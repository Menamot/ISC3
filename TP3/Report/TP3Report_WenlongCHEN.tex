\documentclass[11pt]{article}
\usepackage{amssymb,amsmath,amsthm}
\usepackage{bm, mathrsfs}
\usepackage{graphicx}
\usepackage{geometry}
\usepackage{textcomp}
\usepackage{hyperref}
\usepackage{ragged2e}
\usepackage{float}
\graphicspath{ {./images/} }
\newtheorem{remark}{Remarque}
\newcommand{\bx}{\bm{x}}

\title{ISC3, Fall 2022 (A22) \\
 Computer works report TP 3, 03/10/2022}
\author{Wenlong CHEN}
\date{\today}

\begin{document}
    \maketitle
    \section*{Exercise 1}
    Let $(x,y,z) \rightarrow F(x,y,z)$ be the mapping defined by
    $$
    F(x,y,z)=
    \begin{bmatrix}
        (x-2)^2-1 \\
        (y-z-3)^2 \\
        (z+1)^2-1
    \end{bmatrix}
    $$
    Write a Scilab function Fout=F(xvec) that implements the mapping \textbf{F}, and a Scilab function Jout=FJac(xvec) that computes the Jacobian matrix of \textbf{F}.\\
    Solution : \\
    Sur la base de la définition de la matrice de Jacobi et de la fonction F, nous pouvons écrire la matrice de Jacobi de la fonction F
    $$
    \begin{bmatrix}
        2*(x-2) & 0 & 0 \\
        0 & 2*(y-z-3) & -2*(y-z-3) \\
        0 & 0 & 2*(z+1)
    \end{bmatrix}
    $$
    Code pour cette question : 
    \begin{verbatim}
        function Fout = F(xvec)
            Fout = zeros(1,3)
            Fout(1) = (xvec(1) - 2)^2 - 1
            Fout(2) = (xvec(2) - xvec(3) - 3)^2
            Fout(3) = (xvec(3) + 1)^2 - 1
        endfunction

        function Jout = FJac(xvec)
            Jout = zeros(3,3)
            Jout(1,1) = 2 * (xvec(1) - 2)
            Jout(2,2) = 2 * (xvec(2) - xvec(3) - 3)
            Jout(2,3) = -2 * (xvec(2) - xvec(3) - 3)
            Jout(3,3) = 2 * (xvec(3) + 1)
        endfunction
    \end{verbatim}
    ~\\

    \section*{Exercise 2}
    Implement the Newton method that numerically solve $F(x) = 0$. Use $x_0 =(0,0,0)$ as initial guess.\\
    Solution : \\
    Pour une erreur de $10^{-5}$, seules onze itérations sont utilisées pour obtenir $x=(1,3,0)$.
    Code pour cette question :
    \begin{verbatim}
        function [x,k]=Newton(x0,Fout,Jout,err) // k est le Nombre d'itérations
            k=1;x=x0;
            while abs(norm(F(x))) >err
                x = x+(-Jout(x)\Fout(x)')
                k=k+1
            end
        endfunction

        [x,k] = Newton(zeros(3,1),F,FJac,10^(-5))
    \end{verbatim}
    ~\\

    \section*{Exercise 3}
    Consider an articulated arm made of two rods of respective length 4 and 3. The articulated arm is fixed at the origin $O(0,0)$. The midpoint $A$ that binds the two rods has coordinates $A(x,y)$. The free endpoint is denoted by $B$.\\
    (a) Find $(x,y)$ such that B = $(3,3)$.\\
    
\end{document}